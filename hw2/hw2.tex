\documentclass[11pt]{article}

\usepackage{../analysis}

\begin{document}

\coverpage{2}

% hw problem 1 -----------------------------------------------------------------

\begin{exercise}{6.2.4}{6}
    \problem{
        Prove that if $f$ is Riemann integrable on $[a,b]$ and $g(x) = f(x)$ for every $x$ except for a finite number, then $g$ is Riemann integrable.
    }
    \proof{
        We must show that $g$ is Riemann integrable.
        The function $g$ is Riemann integrable if there exists sequence of partitions $P_j$ and a real number $\int _b^a g(x) dx$ for which $S(g, P_j) \to \int _b^a g(x) dx$ as $j \to \infty$ for every choice of Cauchy Sums $S(g, P_j)$.
        We already know that this condition holds for the function $f$.
        We will just show that the difference between any Cauchy Sum $S(f, P_j)$ and $S(g, P_j)$ becomes arbitrarily small as $j \to \infty$. 
        Then the same conditions will hold for $g$. \parspace
        Given a partition $P_j$, let $|P_j|$ denote the maximum interval length in $P_j$.
        Now, since the number of difference between $f$ and $g$ is finite, label the values $x_1, x_2, ..., x_n$ such that $g(x_i) \neq f(x_i)$.
        Then we must have $| S(f, P_j) - S(g, P_j) | \leq \sum _{i=1}^n |f(x_i) - g(x_i)| |P_j|$ (the differences between the function times the maximum interval length).
        Now let $D = \max _i | f(x_i) - g(x_i) |,$ then $| S(f, P_j) - S(g, P_j) | \leq \sum _{i=1}^n D |P_j| = nD|P_j|$.
        Since $nD$ is finite, as $j \to \infty$, $|P_j| \to 0$ so the magnitude of the difference is 0 as $j \to \infty$.
        Then $\int _b^a f(x) dx = \int _b^a g(x) dx$ which implies that $g$ is Riemann integrable.
    }
\end{exercise}

% hw problem 2 -----------------------------------------------------------------

\begin{exercise}{6.2.4}{9}
    \problem{
        If $f$ is a Riemann integrable function on $[a,b]$ prove that $F(x) = \int_a^x f(t) dt$ satisfies a Lipschitz condition.
    }
    \proof{
        The function $F(x)$ is Lipschitz if there exists a natural number $M$ such that $| F(x) - F(x_0) | \leq M |x - x_0 |$ for all $x, x_0 \in [a,b]$.
        Plugging in $x$, we must have $| \int _a^x f(t) dt - \int _a^{x_0} | \leq M | x - x_0 |$ for all $x, x_0 \in [a,b]$. 
        Choose $M = \sup_{y \in [a,b]} f(y)$. \parspace 
        Now we will show that $\int _a^x f(t) dt - \int _a^{x_0} f(t) dt = \int _{x_0}^x f(t) dt$.
        We consider three cases.
        Case $x = x_0:$ then $\int _a^x f(t) dt - \int _a^{x_0} f(t) dt = 0 = \int _{x_0}^x f(t) dt$.
        Case $x < x_0:$ then $\int _a^x f(t) dt - \int _a^{x_0} f(t) dt = \int _a^x f(t) dt - [\int _a^x f(t) dt + \int _x^{x_0} f(t) dt ] = - \int _x^{x_0} f(t) dt = \int _{x_0}^x f(t) dt$.
        Case $x > x_0$: then $\int _a^x f(t) dt - \int _a^{x_0} f(t) dt = [\int _a^{x_0} f(t) dt + \int _{x_0}^x f(t) dt] - \int _a^{x_0} f(t) dt = \int _{x_0}^x f(t) dt$. \parspace
        So now have $| \int _a^x f(t) dt - \int _a^{x_0} f(t) dt | = | \int _{x_0}^x f(t) dt |$.
        Then we immediately obtain $| \int _{x_0}^x f(t) dt | \leq M_0 |x - x_0|$ where $M_0 = \sup f(x)$ on $[x_0, x]$ (or $[x, x_0]$ if $x < x_0$).
        Then $M_0 |x - x_0| \leq \sup _{x \in [a,b]} f(x) |x - x_0| = M |x - x_0|$.
        So we have just shown that $| F(x) - F(x_0) | \leq M |x - x_0|$ for our chosen $M$.
        Therefore $F(x)$ is Lipshitz continuous.
    }
\end{exercise}

% hw problem 3 -----------------------------------------------------------------

\begin{exercise}{6.2.4}{10}
    \problem{
        If $f$ is Riemann integrable on $[a,b]$ and continuous at $x_0$, prove that $F(x) = \int_a^x f(t) dt$ is differentiable at $x_0$ and $F'(x_0) = f(x_0)$.
        Show that if $f$ has a jump discontinuity at $x_0$, then $F$ is not differentiable at $x_0$.
    }
    \proof{
        
    }
\end{exercise}

\end{document}
