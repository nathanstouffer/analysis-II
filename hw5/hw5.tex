\documentclass[11pt]{article}

\usepackage{../analysis}

\begin{document}

\coverpage{5}

% hw problem 1 -----------------------------------------------------------------

\begin{exercise}{7.4.5}{2}
    \problem{
        If $f$ is analytic in a neighborhood of $x_0$ and $f( x_0) = 0$, show that $f(x)/(x - x_0)$ is analytic in the same neighborhood.
    }
    \proof{
        So we know that $f$ is analytic in some neighborhood $(x_0 - 1/n, x_0 + 1/n)$.
        Let $y$ be any fixed point in $(x_0 -1/n, x_0 + 1/n)$.
        Since $f(x)$ is analytic at $y$, there exists a power series expansion $f(x) = \sum _{n=0}^\infty a_n (x - y)^n$ about $y$.
        By the uniqueness of power series,
        $ \sum _{n=0}^\infty a_n (x - y)^n =
        \sum_{n=0}^\infty b_n (x - x_0)^n
        = \sum _{n=0}^\infty \frac{f^{(n)} (x_0)}{n!} (x - x_0)^n$.
        Then $b_n = \frac{f (x_0)}{0!} = \frac{0}{1} = 0$ and
        $$ f(x) = \sum _{n=0}^\infty \frac{f^{(n)} (x_0)}{n!} (x-x_0)^n
        = \sum _{n=1}^\infty \frac{f^{(n)} (x_0)}{n!} (x-x_0)^n$$
        We can divide both sides by $x - x_0$, leaving
        $$ \frac{f(x)}{x-x_0} = \frac{1}{x-x_0} \sum _{n=1}^\infty \frac{f^{(n)} (x_0)}{n!} (x - x_0)^n
        = \sum_{n=1}^\infty \frac{f^{(n)} (x_0)}{n!} (x - x_0)^{n-1} $$
        Thus $f(x)/(x-x_0)$ has a power series expansion $P$ about $x_0$.
        Let's now check that the power series $P$ has the same radius of convergence as the power series expansion of $f(x)$ about $x_0$.
        The power series expansion of $f(x)$ about $x_0$ has radius of convergence $1/R$.
        For $P$, the radius of convergence is $\lim \sup _{n \to \infty} \left( \frac{f^{(n)} (x_0)}{n!} \right) ^{1/n} $, but this is the exact same expression as the expansion of $f(x)$ so we know its value is also $1/R$.
        Therefore, the two expansions have the same radius of convergence which implies that that $f(x)/(x-x_0)$ is analytic in the same neighborhood.
    }
\end{exercise}

% hw problem 2 -----------------------------------------------------------------

\begin{exercise}{7.4.5}{6}
    \problem{
        Prove that if $f(x)$ is analytic on $(a,b)$, then $F(x) = \int _c ^x f(t) dt$ is also analytic on $(a,b)$, where $c$ is any point in $(a,b)$.
    }
    \proof{
        Fix $c \in (a,b)$.
        Then since $f$ is analytic on $(a,b)$, we have $f(x) = \sum _{n=0}^\infty a_n (x - c)^n$ where $a_n = \frac{f^{(n)} (c)}{n!}$.
        Then since $F(x) = \int _c^x f(t) dt$,
        $$ F(x)
        = \int _c^x \sum _{n=0}^\infty \frac{f^{(n)} (c)}{n!} (t - c)^n dt
        = \left. \left( \sum _{n=0}^\infty \frac{f^{(n)} (c)}{(n+1)!} (t-c)^{n+1} \right) \right| _c^x $$
        Then evaluating,
        $$ F(x) = \sum _{n=0}^\infty \frac{f^{(n)} (x_0)}{(n+1)!} (x - c)^{n+1} - \sum _{n=0}^\infty \frac{f^{(n)} (c)}{(n+1)!} (c - c)^{n+1}$$
        But the second term is 0 so we are left with $F(x) = \sum _{n=0}^\infty \frac{f^{(n)} (c)}{(n+1)!} (x - c)^{n+1}$.
        Since $c$ was arbitrary in $(a,b)$, $F(x)$ has a power series expansion about any point in $(a,b)$, thus $F(x)$ is analytic.
    }
\end{exercise}

% hw problem 3 -----------------------------------------------------------------

\begin{exercise}{7.4.5}{7}
    \problem{
        Compute the power-series expansion of the $f(x) = x^2 / (1 - x^2)$ about $x = 0$.
    }
    \proof{
        We wish to find the power series expansion for $x^2 / (1-x^2)$ about $x=0$.
        Let $u = x^2$.
        Then
        $$ \frac{x^2}{1-x^2} = \frac{u}{1-u} = u \frac{1}{1-u}$$
        Now we know the power series for $1/(1-u) = \sum _{n=0}^\infty u^n$
        so
        $$ u \frac{1}{1-u} = u \sum _{n=0}^\infty u^n = \sum _{n=0}^\infty u^{n+1} $$
        Since, we have $u = x^2$, $\sum _{n=0}^\infty u^{n+1} = \sum _{n=0}^\infty (x^2)^{n+1} = \sum _{n=0}^\infty x^{2n+2}$
        so we have the power expansion for $f(x) = x^2 / (1 - x^2)$ about $x = 0$.
        Note this is valid only for $|u| = |x^2| = x^2 < 1$.
    }
\end{exercise}

% hw problem 4 -----------------------------------------------------------------

\begin{exercise}{7.4.5}{8}
    \problem{
        Compute the radius of convergence of the following power series: \\
        \indent a. $\sum n^4 /n! x^n$ \\
        \indent b. $\sum \sqrt{n} x^n$ \\
        \indent c. $\sum n^2 2^n x^n$
    }
    \proof{
        \\ \indent a. Here $1/R = \lim \sup _{n \to \infty} (n^4/ n!)^{1/n} = \lim _{n \to \infty} (n^4)^{1/n} \lim _{n \to \infty} (1/n!)^{1/n}$.
        Since $n^4$ is a polynomial in $n$ we know $\lim _{n \to \infty} (n^4)^{1/n} = 1$ and it was proven in the textbook that $\lim _{n \to \infty} (1/n!)^{1/n} = 0$.
        Since the RHS is 0, $R = + \infty$. \\\\
        \indent b. Again $1/R = \lim \sup _{n \to \infty} (\sqrt{n})^{1/n} = (\lim \sup _{n \to \infty} n^{1/n} )^{1/2} = 1^{1/2} = 1$.
        So $R = 1$. \\\\
        \indent c. As always, $1/R = \lim \sup _{n \to \infty} (n^2 2^n)^{1/n} = \lim _{n \to \infty} (n^2)^{1/n} \lim _{n \to \infty} (2^n)^{1/n} = 1 \lim _{n \to \infty} 2 = 2$ which means $R = 1/2$.
    }
\end{exercise}

\end{document}
