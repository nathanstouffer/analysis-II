\documentclass[11pt]{article}

\usepackage{../analysis}

\begin{document}

\coverpage{1}

% hw problem 1 -----------------------------------------------------------------

\begin{exercise}{6.1.5}{3}
    \problem{
        Derive the integration of the derivative theorem from the differentiation of the integral theorem.
    }
    \proof{
        So we must show that $\ddx \int_a^x g(t) dt = g(x)$ for all continuous functions $g$ on $[a,b]$ with $a \leq b$ implies that $\int_a^b f'(x) dx = f(b) - f(a)$ for all $f \in C^1$ on $[a,b]$.
        Take $F(x) \in C^1[a,b]$, then $F$ has a continuous derivative $F' = f$ defined on $[a,b]$.
        By the differentiation of the integral theorem, we have $\ddx \int_a^x f(t) dt = f(x)$.
        Then $f(x) = F'(x) = \ddx F(x)$ so we must have $F(x) = \int_a^x F'(t) dt$ an arbitrary $C^1$ function on $[a,b]$. \parspace
        Then we can compute the value of $F(b) - F(a) = \int_a^b F'(x) dx - \int_a^a F'(x) dx = \int_a^b F'(x) dx + 0$.
        So we have shown that $F(b) - F(a) = \int_a^b F'(x) dx$ holds for any $C^1$ function defined on $[a,b]$, which is the integration of the derivative theorem.
    }
\end{exercise}

% hw problem 2 -----------------------------------------------------------------

\begin{exercise}{6.1.5}{4}
    \problem{
        Prove the integral mean value theorem: if $f$ is continuous on $[a,b]$ then there exists $y$ in $(a,b)$ such that $\int_a^b f(x) dx = (b-a)f(y)$.
    }
    \proof{
        Suppose we have some continuous function $f$ defined on $[a,b]$.
        Then let $F(x) = \int_a^x f(t) dt$, we know $F(x) \in C^1[a,b]$ and $F' = f$ by the differentiation of the integral theorem.
        Since $F(x)$ is differentiable on $[a,b]$, the mean value theorem tells us there exists $y \in (a,b)$ such that
        $$F'(y) = f(y) = \frac{F(b) - F(a)}{b-a} = \frac{\int_a^b f(x)dx - \int_a^a f(x)dx}{b-a} = \frac{\int_a^b f(x)dx - 0}{b-a} = \frac{\int_a^b f(x)dx}{b-a}$$
        So we have $f(y) = \frac{\int_a^b f(x)dx}{b-a}$ and we can multiply both sides by $b-a$ to obtain $\int_a^b f(x)dx = (b-a)f(y)$ for some $y \in (a,b)$ as desired.
    }
\end{exercise}

% hw problem 3 -----------------------------------------------------------------

\begin{exercise}{6.1.5}{8}
    \problem{
        Let $f$ be a $C^1$ function on the line, and let $g(x) = \int_0^1 f(xy)y^2 dy$.
        Prove that $g$ is a $C^1$ function and establish a formula for $g'(x)$ in terms of $f$.
    }
    \proof{
        First define $h(x,y) = f(xy) y^2$ then $g(x) = \int_0^1 h(x,y) dy$.
        Since $f(xy)$ and $y^2$ are continous functions, Theorem 6.1.8 tells us that $\int_0^1 h(x,y) dy$ is a continous function.
        Further, the $g(x)$ must be $C^1$ for its derivative is continous.
        Then the conditions for Theorem 6.1.7 are met (since the constant functions 0 and 1 are certainly $C^1$) so we can give the derivative formula:
        $$ g'(x) = \int_0^1 \frac{\partial h}{\partial x}(x,y) dy = \int_0^1 f'(xy) y^3 dy $$
    }
\end{exercise}

% hw problem 4 -----------------------------------------------------------------

\begin{exercise}{6.1.5}{10}
    \problem{
        For a continuous, positive function $w(x)$ on $[a,b]$, define the weighted average operator $A_w$ to be
        $$ A_w(f) = \int_a^b f(x)w(x) dx / \int_a^b w(x) dx $$
        for continous functions $f$.
        Prove that $A_w$ is linear and lies between the maximum and minimum values of $f$.
    }
    \proof{
        First we prove that $A_w$ is linear.
        $A_w$ is linear if $A_w(c_1f + c_2 g) = c_1 A_w(f) + c_2 A_w(g)$ for $c_1, c_2 \in \R$ and continuous functions $f,g$ on $[a,b]$.
        Towards proving linearity, let $f,g$ be continuous functions on $[a,b]$ and $c_1, c_2 \in \R$.
        Then, using properties of integrals and functions,
        \begin{align*}
            A_w(c_1 f + c_2 g) &= \dfrac{\int_a^b [c_1 f + c_2 g](x)w(x)}{\int_a^b w(x) dx} \\
                               &= \dfrac{\int_a^b c_1 f(x) w(x) + c_2 g(x) w(x) dx}{\int_a^b w(x) dx} \\
                               &= \dfrac{\int_a^b c_1 f(x) w(x) dx + \int_a^b c_2 g(x) w(x) dx}{\int_a^b w(x) dx} \\
                               &= \dfrac{c_1 \int_a^b f(x) w(x) dx + c_2 \int_a^b g(x) w(x) dx}{\int_a^b w(x) dx} \\
                               &= \dfrac{c_1 \int_a^b f(x) w(x) dx}{\int_a^b w(x) dx} + \dfrac{c_2 \int_a^b g(x) w(x) dx}{\int_a^b w(x) dx} \\
                               &= c_1 A_w(f) + c_2 A_w(g)
        \end{align*}
        So $A_w$ is linear.
        Now we must show that $A_w(f)$ lies in between the maximum and minimum values of $f$.
        Let $F^+ = \sup _{x \in [a,b]} f(x)$ and $F^- = \inf _{x \in [a,b]} f(x)$, so we must show that $A_w(f) \in [F^-, F^+]$ which is equivalent to $F^- \leq A_w(f) \leq F^+$.
        Since $\int_a^b w(x) dx$ is a postive number, let $W = \int_a^b w(x) dx$.
        If multiply the desired inequality by $W$, we must equivalently show that $WF^- \leq \int_a^b f(x) w(x) dx \leq WF^+$. \parspace
        Now it is certainly true that $\int_a^b f(x) w(x) dx \leq \int_a^b F^+ w(x) dx$, but then $F^+$ is a constant so $ \int_a^b F^+ w(x) dx = F^+ \int_a^b w(x) dx = F^+ W $ so $ \int_a^b f(x) w(x) dx \leq WF^+$.
        Similarly, $\int_a^b f(x) w(x) dx \geq \int _a^b F^- w(x) dx = F^- \int_a^b w(x) dx = F^-W$.
        Then we have $WF^- \leq \int_a^b f(x) w(x) dx \leq WF^+$, which was the goal.
    }
\end{exercise}

\end{document}
