\documentclass[11pt]{article}

\usepackage{../analysis}

\begin{document}

\coverpage{3}

% hw problem 1 -----------------------------------------------------------------

\begin{exercise}{6.3.2}{1}
    \problem{
        For which values of $a$ and $b$ does the improper integral $\int _0^{1/2} x^a | \log x | ^b dx$ exist?
    }
    \proof{
        We consider three cases for the value of $a$: $a < -1, a = -1, a > -1$.
        Beginning with $a > -1$ then we know there exists some $\epsilon > 0$ such that $a - \epsilon > -1$.
        Then $\int _0^{1/2} x^a | \log x | ^b dx = \int _0^{1/2} x^{a-\epsilon} x^\epsilon | \log x | ^b dx $.
        Then $\lim _{x \to 0^+} x^\epsilon |\log x|^b = 0$ so we need only worry about the convergence of $\int _0^{1/2} x^{a-\epsilon}$.
        But $a - \epsilon > -1$ so we know this integral to converge. \parspace
        Now consider $a < -1$ then there exists some $\epsilon > 0$ sucht that $a + \epsilon < -1$.
        Then $\int _0^{1/2} x^a | \log x | ^b dx = \int _0^{1/2} x^{a+\epsilon} x^{-\epsilon} | \log x | ^b dx $.
        Then $\lim _{x \to 0^+} x^{- \epsilon} | \log x | ^b = \infty$ so we are toast in this case. \parspace
        When $a = -1$, we have $\int _0^{1/2} x^a | \log x | ^b dx = \int _0^{1/2} \frac{| \log x | ^b}{x} dx $.
        Taking $\log$ to be the natural logarithm, we have $\int _0^{1/2} \frac{| \ln x | ^b}{x} dx = \frac{| \ln x | ^{b+1}}{b+1} \Bigr\rvert _0^{1/2}$ (even when $b=0$) which diverges to $-\infty$ when 0 is plugged in.
    }
\end{exercise}

% hw problem 2 -----------------------------------------------------------------

\begin{exercise}{7.2.4}{1}
    \problem{
        Give and example of two convergent series $\sum _{k=1}^\infty x_k$ and $\sum _{k=1}^\infty y_k$ such that $\sum _{k=1}^\infty x_k y_k$ diverges.
        Can this happen if one of the series is absolutely convergent?
    }
    \proof{
        For the infinite series $\sum x_k, \sum y_k$, let $x_k = (-1)^k / \sqrt{k}$ and $y_k = (-1)^{k+1} / \sqrt{k}$.
        Note that $- \sum x_k = \sum y_k$ so we only need to show that convergence of one series.
        Let's show $\sum x_k$.
        If we set $A_k = 1/ \sqrt{k}$ then $\sum x_k = \sum _{k=1}^\infty (-1)^k A_k$ where $A_k$ monotonically converges to 0.
        Then we can apply Theorem 7.2.5 to say that $\sum x_k$ is convergent.
        So each of $\sum x_k$ and $\sum y_k$ converge, what about their pairwise product?
        $$ \sum _{k=1}^\infty x_k y_k
        = \sum _{k=1}^\infty ((-1)^k / \sqrt{k}) * ((-1)^{k+1} / \sqrt{k})
        = \sum _{k=1}^\infty (-1)^{2k+1} / k
        = \sum _{k=1}^\infty -1 / k
        = - \sum _{k=1}^\infty 1/k$$
        But this is the negative of the harmonic series, which we know diverges so we have given an example of the desired series. \parspace
        Now suppose that one of the series is absolutely convergent.
        Without loss of generality, pick $\sum x_k$.
        Then $\sum _{k=1}^\infty |x_k|$ converges.
        Additionally, we know $\sum y_k$ converges.
        Since $\sum y_k$ converges, we know $\lim _{k \to \infty} y_k = 0$.
        Then for all $1/n$ there exists and $m$ for which $|y_j| < 1/n$ for all $j \geq m$.
        Choose $n = 1$ and let $m_1$ be the index that satisfies the Cauchy criterion.
        Certainly the partial sum $\sum _{k=1}^{m_1-1} x_k y_k $ converges.
        What about $\sum _{k=m_1}^\infty x_k y_k$?
        Choose any $N > m_1$, then we have $ \left| \sum _{k=m_1}^N x_k y_k \right| \leq \sum _{k=m_1}^N | x_k y_k | = \sum _{k=m_1}^N |x_k| |y_k| $
        by the triangle inequality.
        But since $|y_j| < 1$ for all $j \geq m_1$ so $\sum _{k=m_1}^N |x_k| |y_k| \leq \sum _{k=m_1}^N |x_k|$.
        Then we can the limit as $N \to \infty$ and the non-strict inequality is preserved.
        But we already know $\sum |x_n|$ converges so $\sum x_n y_n$ converges.
    }
\end{exercise}

% hw problem 3 -----------------------------------------------------------------

\begin{exercise}{7.2.4}{2}
    \problem{
        State a contrapositive form of the comparison test that can be used to show divergence of a series.
    }
    \proof{
        Contrapositive: For infinite series $\sum _{k=1}^\infty x_k$ and $\sum _{k=1}^\infty y_k$ with non-negative $x_k$ and $x_k \leq |y_k|$, we can say that if $\sum _{k=1}^\infty x_k$ diverges that $\sum _{k=1}^\infty y_k$ is divergent.
    }
\end{exercise}

% hw problem 4 -----------------------------------------------------------------

\begin{exercise}{7.2.4}{4}
    \problem{
        Prove the ratio test.
        What does this tell you if $\lim _{n \to \infty} |x_{n+1} / x_n|$ exists?
    }
    \proof{
        The ratio test claims that if $|x_{n+1} / x_n | < r$ for all sufficiently large $n$ and some $r < 1$, then $\sum x_n$ converges absolutely while if $|x_{n+1} / x_n| \geq 1$ for all sufficiently large $n$, then $\sum x_n $ diverges. \parspace
        First we prove the absolutely convergent result.
        Consider some infinite series that satisfies $|x _{n+1} / x_n | < r$ for all sufficiently large $n$.
        Then there exists a natural number $K$ such that for all $n > K$ we have $| x_{n+1} / x_n | < r$.
        Then the infinite series $\sum x_n = \sum _{n=1}^K x_n + \sum _{n=K+1}^\infty x_n$.
        Letting $y_n = x_{n+K}$, the convergence of $\sum x_n$ rests on $\sum _{n=K+1}^\infty x_n = \sum _{n=1}^\infty y_n$. \parspace
        By supposition, we have (for all $n > K$) $| x_{n+1} / x_n | < r$.
        Because of how we defined $y_n$, this means we have $|y_{n+1} / y_n| < r$ for all $n$.
        This implies that $| y_{n+1} | < r | y_n |$, which, in turn, means that $|y_n| < r^{n-1} |y_1|$ for $n \geq 2$.
        Then let $z_n = r^{n-1} y_1$.
        We know $\sum _{n=1}^\infty z_n$ converges because $\sum _{n=1}^\infty z_n = \sum _{n=1}^\infty  r^{n-1} y_1 = y_1 \sum _{n=1}^\infty  r^{n-1} $ converges since $|r| < 1$.
        Now we can use the comparison test for $\sum z_n$ and $\sum y_n$ to say that $\sum y_n$ converges absolutely, which implies the absolute convergence of $\sum x_n$. \parspace
        We now show that if $| x_{n+1} / x_n | \geq 1$ for all sufficiently large $n$ then $\sum x_n$ diverges.
        Suppose we have such a summation $\sum x_n$.
        Then there exists some $K$ such that $| x_{n+1} / x_n | \geq 1$ for all $n > K$.
        But then we have $|x_{n+1}| \geq |x_n|$ for all $n > K$.
        This certainly cannot satisfy $\lim _{n \to \infty} x_n = 0$, a necessary condition for convergence.
        Therefore, $\sum x_n$ diverges. \parspace
        If we know that $\lim _{n \to \infty} |x_{n+1} / x_n|$ exists, then I'm not sure what we can say.
        It seems like the series may converge or diverge.
    }
\end{exercise}

\end{document}
