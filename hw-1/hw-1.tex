\documentclass[11pt]{article}

\usepackage{../analysis}

\begin{document}

\coverpage{1}

% hw problem 1 -----------------------------------------------------------------

\begin{exercise}{1.1.3}{2}
    \problem{
        statement for problem 1
    }
    \proof{
    }
\end{exercise}

% hw problem 2 -----------------------------------------------------------------

\begin{exercise}{1.2.3}{1}
    \problem{
        Prove that every subset of $\N$ is either finite or countable.
        Conclude from this that there is no infinite set with cardinality less than that of $\N$.
    }
    \proof{
    }
\end{exercise}

% hw problem 3 -----------------------------------------------------------------

\begin{exercise}{1.2.3}{3}
    \problem{
        Prove that the rational numbers are countable.
    }
    \proof{
    }
\end{exercise}

% hw problem 4 -----------------------------------------------------------------

\begin{exercise}{1.2.3}{4}
    \problem{
        Show that if a countable subset is removed from an uncountable set, the remainder is still uncountable.
    }
    \proof{
    }
\end{exercise}

% hw problem 5 -----------------------------------------------------------------

\begin{exercise}{1.2.3}{5}
    \problem{
        Let $ A_1, A_2, A_3, ... $ be countable sets, and let their Cartesian product $ A_1 \times A_2 \times A_3 \times \cdot \cdot \cdot $ be defined to be the set of all sequences $( a_1, a_2, ... )$ where $ a_k $ is an element of $ A_k $.
        Prove that the Cartesian product is uncountable.
        Show that the same conclusion holds if each of the sets $ A_1, A_2, ... $ has at least two elements.
    }
    \proof{
    }
\end{exercise}

\end{document}
